\documentclass{scrarticle}
\usepackage[utf8]{inputenc}
\usepackage[T2A]{fontenc}
\usepackage[russian]{babel}
\usepackage{amssymb}
\usepackage{amsmath}
\usepackage{amsthm}
\usepackage{listings}
\usepackage{hyperref}
\newtheorem{theorem}{Теорема}
\newtheorem{corollary}{Следствие}[theorem]
\newtheorem{lemma}[theorem]{Лемма}
\newcommand{\divisible}{\mathop{\raisebox{-2pt}{\vdots}}}
\title{Коллоквиум по теории чисел.}
\author{Титилин Александр}
\date{}
\begin{document}
    \maketitle
    \section{Делимость}
    \[
    a , b \in \mathbb{Z}
    .\] 
    \[
    a \divisible b := \exists \alpha \in \mathbb{Z} : a = \alpha b
    .\] 
    \begin{theorem} \label{1}
        \[
        a,b,c\in \mathbb{Z}
        .\] 
        \[
        a \divisible b \land b \divisible c \implies a \divisible c
        .\] 
    \end{theorem}
    \begin{proof}
        \[
        \exists \alpha , \beta \in \mathbb{Z}
        .\] 
        \[
        a = \alpha b \land b = \beta c \implies a = \alpha \beta c
        .\] 
    \end{proof}
    \begin{theorem} \label{2}
        \[
        a,b,c \in \mathbb{Z}
        .\] 
        \[
        a \divisible c \land b \divisible c \implies
        (a \pm b) \divisible c
        .\] 
    \end{theorem}
    \begin{proof}
        \[
        \exists \alpha , \beta \in \mathbb{Z}
        .\] 
        \[
        a = \alpha c \land b = \beta c
        .\] 
        \[
        a \pm b = c(\alpha \pm \beta)
        .\] 
    \end{proof}
    \begin{theorem} \label{3}
        Каждое натуральное число больше единицы делится хотя бы на одно простое число.
    \end{theorem}
    \begin{proof}
        Рассуждаем по индукции. Для 2 теорема верна, так как 2 простое число. Теперь предположим, что теорема верна для всех чисел меньше $n$. Если  $n$ простое, то теорема верна, иначе $n = ab$,  $a < n$.  $a$ делится на простое число. n делится на простое по теореме \ref{1}.
    \end{proof}
    \begin{theorem}
        Существует бесконечно много простых чисел.
    \end{theorem}
    \begin{proof}
        Пусть простых чисел конечное число $p_1,p_2\dots p_k$ 
        Рассмотрим число $n = p_1 p_2 p_3 \dots p_k + 1$. n не простое число и не делится ни на одно простое, получили противоречие.
    \end{proof}
    \begin{theorem}
        Каждое натуральное число $> 1$ число может быть единсвенным образом записано в виде произвдения степеней простых чисел.
    \end{theorem}
    \begin{proof}

        Существование. $n = 2$ умеет записывать таким образом, так как оно само является простым числом. Теперь предположим, что умеем так раскладывать все натуральные числа от 1 до  $n$ таким образом. Если  $n$ простое, то разложение очевидно. Иначе $n = ab, a < n, b < n$

        Единственность. Пусть есть два разных разложения, сравним их сократив их общие множители
        \[
            p_1^{r_1}p_2^{r_2} \dots p_s^{r_s} = q_1^{t_1}q_2^{t_2} \dots q_u^{t_u}
        .\] 
        Правая часть делится на $q_1$. $q_1$ взаимно просто с любым $p$ по теоремe \ref{11} левая часть равенства на  $q_1$ не делится.
    \end{proof}
    \section{НОД}
    Если $a \divisible c \land b \divisible c$. $c$ называется общим делителем.
    \begin{lemma}
        \[
        \forall a , b \in \mathbb{Z}
        .\] 
        \[
            \gcd{( a,b )} = \gcd{( a-b,b )}
        .\] 
    \end{lemma}
    \begin{proof}
        \[
        a , b \in \mathbb{Z}
        .\] 
        \[
        a \divisible c \land b \divisible c \implies a - b \divisible c
        .\] 
    \end{proof}
    \begin{theorem}
        \[
            m \divisible n \implies \gcd{(m,n)} = n
        .\] 
    \end{theorem}
    \begin{theorem}
        \[
            \gcd{(am,an)} = |a| \gcd{( m,n )}
        .\] 
    \end{theorem}
    \section{Деление с остатком}
    \begin{theorem}
        \[
            a ,b \in \mathbb{Z}, b \neq 0
        .\] 
        \[
            \exists! q , r \in \mathbb{Z}, r = 0.. |b| - 1
        .\] 
        \[
            a = qb + r, r = 0\dots |b| -1
        .\] 
    \end{theorem}
    \begin{proof}
       Единственность.
       \[
       a = b q_1 + r_1
       .\] 
       \[
       a = b q_2 + r_2
       .\] 
       \[
           b(q_1 - q_2) = r_1 - r_2
       .\] 
       \[
       |b (q_1 - q_2) | = |r_1 - r_2|
       .\] 
    \end{proof}
    \begin{theorem}
        \[
        a , b \in \mathbb{Z}
        .\] 
        \[
            d = \gcd{a,b}
        .\] 
        \[
        \exists  \alpha , \beta \in \mathbb{Z}
        .\] 
        \[
        \alpha a + \beta b = d
        .\] 
    \end{theorem}
    \begin{proof}
        Рассмотрим множество $ M  = \{ax + by | x,y \in \mathbb{Z} \land ax + by > 0 \}$. $ax \divisible d \land by \divisible d \implies \forall \delta \in M ~ \delta \divisible d$. $d = \min{M}$
    \end{proof}
    \begin{corollary}
        \[
            \gcd{(a,b)} = 1 \iff \exists x,y \in \mathbb{Z} ~  ax + by = 1
        .\] 
    \end{corollary}
    \begin{theorem} \label{11}
        \[
        a , b \in \mathbb{Z}
        .\] 
        \[
            ab \divisible c \land \gcd{(b,c)} = 1 \implies a \divisible c
        .\] 
    \end{theorem}
    \begin{proof}
        \[
        ab = \alpha c
        .\] 
        \[
        bx + ny = 1 \implies ( ab )x + (ay)n = a 
        .\] 
    \end{proof}
    \section{Сравнение по модулю}
    \[
    n \in \mathbb{N}, n \neq 1
    .\] 
    \[
    a , b \in \mathbb{Z}
    .\] 
    \[
        a \equiv b \pmod{n} := a - b \divisible n
    .\] 
    \begin{theorem}
        Если
        \[
            a \equiv b \pmod{n} \land c \equiv d \pmod{n}
        .\] 
        то
        \begin{itemize}
            \item $a + c \equiv b + d \pmod{n}$
            \item  $ac \equiv bd \pmod{n}$
        \end{itemize}
    \begin{proof}
            \[
            a - b = \alpha n
            .\] 
            \[
            c - d = \beta n
            .\] 
        \begin{itemize}
            \item
            \[
                (a + c) - (b + d) = n(\alpha + \beta)
            .\] 
            \item
                \[
                    ac - bd = ac - ad - bd + ad = a(c - d) - d (a - b) = (a\beta)n - (d \alpha)n
                .\] 
        \end{itemize}
    \end{proof}
    \end{theorem}
    \begin{theorem}
        \begin{itemize}
            \item $a \equiv a \pmod{n} , \forall  a\in \mathbb{Z}, n\neq 1 n \in \mathbb{N}$
            \item $a \equiv b \pmod{n} \implies b \equiv a \pmod{n}$
            \item $a \equiv b \pmod{n} \land b \equiv c \pmod{n} \implies a \equiv c \pmod{n}$
        \end{itemize}
    \end{theorem}
        \begin{proof}
            \begin{itemize}
                \item 
                    $a - a = 0 , 0 \divisible n$ 
                \item 
                    \[
                    a - b = pn
                    .\] 
                    \[
                    b - a = -(a - b) = -pn
                    .\] 
                \item
                    \[
                    a - b = \alpha n
                    .\] 
                    \[
                    b - c = \beta n
                    .\] 
                    \[
                    a - c = n(\alpha + \beta)
                    .\] 
            \end{itemize}
        \end{proof}
        \begin{theorem}
            $\gcd{a,n} = 1 \implies \exists x ~  ax \equiv 1 \pmod{n}$
        \end{theorem}
        \begin{proof}
            \[
                \alpha a + \beta n = 1
            .\] 
            \[
                \alpha a - 1 = \beta n
            .\] 
        \end{proof}
        \begin{theorem}
            $\gcd{a,n} = 1 \land a*b \equiv a * c \pmod{n} \implies b \equiv c \pmod{n}$
        \end{theorem}
        \section{Классы вычетов}
        $[a]_n$ = $\{b \mid b \in \mathbb{Z} , b \equiv a \pmod{n}\}$ -- класс вычетов $a$ по модулю  $n$. ($a = 0..n-1$)
        \begin{theorem}
            Два класса вычетов по одному модулю или совпадают или их пересечение пустое множество.
        \end{theorem}
        \begin{theorem}[Малая теорема Ферма]
            Пусть $p$ простое число,  $a \in \mathbb{Z}$ $a$ не делится на  $p$, тогда
             \[
                 a^{p - 1} \equiv 1 \pmod{p}
            .\] 
        \end{theorem}
        \begin{proof}
            Пусть $a_k$ -- остаток от деления $ka$ на  $p$, где  $k = 1,2,\dots,p-1$. $ak$ не делится на  $p$,c среди $a_k$ нет нулей. Рассмотрим  $\forall n,m \in \{1,2,\dots,p-1\}, n \neq m$. $an - am \neq 0$. Таким образом множество $a_1,a_2,\dots,a_{p - 1}$ совпадает с множеством $\{ 1,2,\dots,p-1 \}$.
            \[
                a_1 a_2 a_3 \dots a_{p-1} = (p - 1)!
            .\] 
            \[
                a^{p - 1} (p - 1)! = a * 2a * \dots * (p - 1)a \equiv (p - 1)! \pmod{p}
            .\] 
        \end{proof}
        $\varphi(n)$ -- функция Эйлера, количество натуральных чисел меньше n взаимнопростых с ним. Если n -- простое, то  $\varphi(n) = n - 1$
        \begin{theorem}[Эйлера]
            a взаимно просто c $n$. Тогда  $a^{\varphi(n)} \equiv 1 \pmod{n}$
        \end{theorem}
        \begin{proof}
            Рассмотрим  $A = \{k_1,k_2, \dots ,k_{\varphi(n)}\}$ -- множество всех чисел, взаимно простых с n. Теперь рассматриваем набор всех остатков от деления $\forall  k \in A$ на $a$. Нулей и одинаковых чисел в таком наборе нет. Этот набор совпадает с $A$
             \[
                 a^{\varphi(n)}k_1k_2\dots k_{\varphi(n)}  = ak_1 * ak_2 * \dots * a*k_{\varphi(n)} \equiv k_1 k_2 \dots k_{\varphi(n)} \equiv 1 \pmod{n}
            .\]
        \end{proof}
        \begin{theorem}
            $\gcd{m,n} = 1 \implies \varphi(mn) = \varphi(m)\varphi(n)$
        \end{theorem}
\end{document}

